\section*{Deployment}

Docker offre un ambiente di containerizzazione che permette di confezionare l'applicazione web e il database MongoDB insieme, garantendo che siano consistenti e facili da distribuire su qualsiasi ambiente, indipendentemente dalle differenze di configurazione.

È stato quindi creato un Dockerfile per buildare l'immagine del server, e un docker-compose.yaml per orchestrare il deployment dell'applicazione web e del database.

Il Dockerfile è multistep per minimizzare il peso dell'immagine finale. La prima immagine è più pesante e contiene tutte le dipendenze necessarie per buildare, mentre la seconda solo quelle necessarie a runtime.

Per lo sviluppo in locale, al posto di docker è conveniente utilizzare yarn dev, a patto di avere un servizio mongo locale che runna sulla porta standard 27017.

