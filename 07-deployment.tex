\section*{Deployment}

Docker offre un ambiente di containerizzazione che permette di confezionare l'applicazione web e il database MongoDB insieme, garantendo che siano consistenti e facili da distribuire su qualsiasi ambiente, indipendentemente dalle differenze di configurazione.

È stato quindi creato un Dockerfile per buildare l'immagine del server, e un docker-compose.yaml per orchestrare il deployment dell'applicazione web e del database.

Il Dockerfile è multistep per minimizzare il peso dell'immagine finale. La prima immagine è più pesante e contiene tutte le dipendenze necessarie per buildare, mentre la seconda solo quelle necessarie a runtime.

Il docker compose è attualmente configurato per l'utilizzo sul server nel quale è stato effettuato il deployment, ma può essere facilmente adattato per l'utilizzo in locale:

\begin{verbatim}
version: '3'

services:

  mongo:
    container_name: mongo
    image: mongo:4.4.18
    volumes:
      - mongodb_data:/data/db
    restart: always
    ports:
      - "27017:27017"

  server:
    container_name: server
    ports:
      - "3033:3033"
    build:
      context: ./
      dockerfile: Dockerfile
    depends_on:
      - mongo
    # environment:
      # - MONGODB=mongodb://mongo

volumes:
  mongodb_data:

\end{verbatim}

