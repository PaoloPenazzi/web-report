\section*{Tecnologie}
Nella seguente sezione verranno analizzate le principali tecnologie utilizzate per lo sviluppo del progetto.
Per ogni tecnologia verrà fornita una breve descrizione e verranno elencati i vantaggi e gli svantaggi che hanno portato alla scelta o meno di utilizzarla.
La scelta delle tecnologie è stata fatta tenendo conto delle conoscenze pregresse dei membri del gruppo e della loro disponibilità a imparare nuove tecnologie, così
come della loro popolarità e della loro adozione in ambito aziendale. Inoltre, le scelte dello stack tecnologico hanno tenuto conto
dei requisiti che ambo gli stakeholders (ricordiamo la dualità di questo progetto, università e azienda) hanno espresso.

\subsection*{TypeScript}

\subsubsection*{Descrizione}
TypeScript è un linguaggio di programmazione open-source sviluppato da Microsoft.
È un super-set di JavaScript, ovvero un linguaggio che estende le funzionalità di un altro linguaggio, in questo caso JavaScript.

\subsubsection*{Vantaggi}
Tra i vantaggi di TypeScript troviamo:
\begin{itemize}
    \item \textbf{TypeScript è JavaScript}: essendo un super-set di JavaScript, TypeScript è compatibile con tutte le librerie JavaScript esistenti;
    \item \textbf{TypeScript è un linguaggio fortemente tipizzato}: questo permette di avere un codice più robusto e di trovare più facilmente errori durante la fase di sviluppo;
    \item \textbf{TypeScript è un linguaggio orientato agli oggetti}: questo permette di avere un codice più modulare e di avere una maggiore astrazione rispetto a JavaScript;
\end{itemize}

\subsubsection*{Svantaggi}
Tra gli svantaggi di TypeScript troviamo:
\begin{itemize}
    \item \textbf{TypeScript è un linguaggio fortemente tipizzato}: l'altro lato della medaglia è che il programmatore deve dichiarare il tipo di ogni variabile, questo può portare ad un aumento della verbosità del codice;
\end{itemize}


\subsection*{React}

\subsubsection*{Descrizione}
React.js è un framework e una libreria open-source sviluppata da Facebook, utilizzata per costruire in modo rapido ed efficiente interfacce utente interattive e applicazioni web con molto meno codice rispetto all'utilizzo di JavaScript "puro".\\

In React, lo sviluppo delle applicazioni viene fatto creando componenti riutilizzabili, simili a blocchi Lego indipendenti. Questi componenti rappresentano parti individuali di un'interfaccia finale che, quando combinati, costituiscono l'intera interfaccia utente dell'applicazione.
\subsubsection*{Vantaggi}
Tra i vantaggi di React troviamo:
\begin{itemize}
    \item \textbf{Componenti Riutilizzabili}: React permette la creazione di componenti personalizzabili e riutilizzabili, migliorando l'efficienza dello sviluppo;
    \item \textbf{Ampio ecosistema}: la comunità al supporto è molto ampia con molte risorse e librerie disponibili per estendere le funzionalità dell'applicazione;
    \item \textbf{Curva di apprendimento}: React è molto semplice da imparare, soprattutto per chi ha già esperienza con JavaScript;
    \item \textbf{SEO frilendly}: è famoso per essere SEO friendly.
\end{itemize}

\subsubsection*{Svantaggi}
Tra gli svantaggi di React troviamo:
\begin{itemize}
    \item \textbf{JSX}: JSX è una sintassi che permette di scrivere codice HTML all'interno di JavaScript, questo può portare ad un aumento della verbosità del codice e c'è chi suggerisce che questo possa portare ad un aumento della ripidità della curva di apprendimento;
    \item \textbf{Documentazione}: la documentazione non è tra le più complete e chiare, complice anche la velocità con cui React evolve.

\end{itemize}


\subsection*{MUI}

\subsubsection*{Descrizione}
MUI è una libreria di componenti React che implementa il Material Design di Google.
Questa libreria è stata scelta per la sua semplicità e per la sua popolarità.
Inoltre è stata utilizzata la libreria Materio (cartella @core del progetto), che include una serie di temi e componenti personalizzati per MUI.

\subsubsection*{Vantaggi}
Tra i vantaggi di MUI troviamo:
\begin{itemize}
    \item \textbf{Semplicità}: MUI è una libreria molto semplice da utilizzare e da configurare;
    \item \textbf{Popolarità}: MUI è una libreria molto popolare, questo permette di trovare facilmente soluzioni ai problemi che si possono incontrare durante lo sviluppo;
    \item \textbf{Documentazione}: MUI ha una documentazione molto chiara e completa;
\end{itemize}

\subsubsection*{Svantaggi}
Tra gli svantaggi di MUI troviamo:
\begin{itemize}
    \item \textbf{Personalizzazione}: La personalizzazione dei componenti alle volte può essere difficile;
\end{itemize}


\subsection*{Docker}

\subsubsection*{Descrizione}
Docker è una piattaforma open-source che permette di creare, testare e distribuire applicazioni in maniera semplice e veloce.
Docker permette di creare dei container, ovvero degli ambienti virtuali isolati, in cui è possibile eseguire le applicazioni. Nel nostro caso 
è stato utilizzato per creare un container in cui è possibile eseguire l'applicativo sviluppato. Attraverso Docker è stato possibile
eseguire il deploy dell'applicativo.

\subsubsection*{Vantaggi}
Tra i vantaggi di Docker troviamo:
\begin{itemize}
    \item \textbf{Semplicità}: Docker permette di creare e gestire i container in maniera molto semplice;
    \item \textbf{Portabilità}: Docker permette di creare dei container che possono essere eseguiti su qualsiasi sistema operativo.
\end{itemize}

\subsubsection*{Svantaggi}
Tra gli svantaggi di Docker troviamo:
\begin{itemize}
    \item \textbf{Risorse}: Docker utilizza molte risorse del sistema;
    \item \textbf{Curva di apprendimento}: Docker è una tecnologia molto ampia e complessa, questo può portare ad una curva di apprendimento molto ripida.
\end{itemize}


\subsection*{MongoDB}

\subsubsection*{Descrizione}
MongoDB è un database NoSQL flessibile che consente di gestire dati non strutturati o semi-strutturati. 

\subsubsection*{Vantaggi}
Tra i vantaggi di MongoDB troviamo:
\begin{itemize}
    \item \textbf{Flessibilità}: MongoDB permette di gestire dati non strutturati o semi-strutturati;
    \item \textbf{Scalabilità}: MongoDB permette di scalare facilmente il database;
    \item \textbf{Agilità di sviluppo}: MongoDB permette di sviluppare applicazioni in maniera molto veloce, soprattutto
    grazie all'ottima integrazione con \textbf{Mongoose}.
\end{itemize}

\subsubsection*{Svantaggi}
Tra gli svantaggi di MongoDB troviamo:
\begin{itemize}
    \item \textbf{Dati duplicati}: può soffrire di problemi di dati duplicati;
    \item \textbf{Risorse}: MongoDB utilizza molte risorse del sistema, come ad esempio la memoria del sistema.
\end{itemize}


\subsection*{Mongoose}

\subsubsection*{Descrizione}
Mongoose è una libreria di ODM (Object-Document Mapping) per
MongoDB e Node.js. Essenzialmente, Mongoose offre un set di strumenti
per semplificare l'interazione con MongoDB, un database NoSQL orientato
ai documenti, tramite una rappresentazione in stile oggetto dei dati.

\subsubsection*{Vantaggi}
Tra i vantaggi di Mongoose troviamo:
\begin{itemize}
    \item \textbf{Agilità di sviluppo}: Mongoose semplifica notevolmente lo sviluppo di applicazioni Node.js che utilizzano MongoDB come database di back-end. Fornisce una struttura chiara e coerente per definire schemi, gestire la validazione dei dati e creare query.
\end{itemize}

\subsubsection*{Svantaggi}
Tra gli svantaggi di Mongoose troviamo:
\begin{itemize}
    \item \textbf{Complessità aggiuntiva}:Mongoose aggiunge una certa complessità all'applicazione a causa della necessità di definire schemi e modelli. Questo potrebbe essere eccessivo per progetti molto semplici o prototipi.
\end{itemize}


\subsection*{Node.js}

\subsubsection*{Descrizione}
Node.js è un ambiente di runtime open-source, multi-piattaforma e orientato agli eventi per l'esecuzione di codice JavaScript lato server.

\subsubsection*{Vantaggi}
Tra i vantaggi di Node.js troviamo:
\begin{itemize}
    \item \textbf{Velocità}: Node.js è molto veloce grazie al suo modello di I/O asincrono;
    \item \textbf{Ampio ecosistema}: la comunità al supporto è molto ampia con molte risorse e librerie disponibili per estendere le funzionalità;
    \item \textbf{JavaScript everywhere}: permette di utilizzare lo stesso linguaggio di programmazione (JavaScript) sia nel frontend che nel backend dell'applicazione, semplificando la condivisione del codice e la sincronizzazione tra il lato client e il lato server.
\end{itemize}

\subsubsection*{Svantaggi}
Tra gli svantaggi di Node.js troviamo:
\begin{itemize}
    \item \textbf{Single threaded}: sebbene sia in grado di gestire molte richieste simultaneamente, Node.js è single threaded, questo può portare ad un aumento del tempo di risposta in caso di richieste molto complesse;
    \item \textbf{Non adatto per calcoli intensivi}: non è la scelta migliore per applicazioni che richiedono calcoli intensivi o operazioni CPU-bound.
\end{itemize}


\subsection*{Next.js}

\subsubsection*{Descrizione}
Next.js è un framework web basato su React che offre un modo semplice ed efficiente per sviluppare applicazioni web reattive e ad alte prestazioni. La sua crescente popolarità nella comunità degli sviluppatori è attribuibile alle sue funzionalità avanzate e alla sua intuitiva facilità d'uso.

\subsubsection*{Vantaggi}
Tra i vantaggi di Next.js troviamo:
\begin{itemize}
    \item \textbf{Server-side rendering}: permette di eseguire il rendering delle pagine lato server, questo permette di avere un sito web più performante e più accessibile;
    \item \textbf{Static site generation}: permette di generare un sito web statico, questo permette di avere un sito web più performante e più sicuro;
    \item \textbf{Zero configuration}: è progettato per essere facile da configurare e richiede meno configurazioni rispetto ad altri framework
\end{itemize}

\subsubsection*{Svantaggi}
Tra gli svantaggi di Next.js troviamo:
\begin{itemize}
    \item \textbf{No Dynamic Routing}: non è ready-to-use per quanto riguarda il dynamic routing, e per questo è necessario Node.js.
\end{itemize}


\subsection*{YARN}

\subsubsection*{Descrizione}
YARN è un package manager per Node.js che permette di gestire le dipendenze di un progetto. È stato scelto per la sua semplicità e per la sua popolarità.

\subsubsection*{Vantaggi}
Tra i vantaggi di YARN troviamo:
\begin{itemize}
    \item \textbf{Semplicità}: YARN è molto semplice da utilizzare;
    \item \textbf{Velocità}: è noto per le sue prestazioni migliorate rispetto a npm;
    \item \textbf{Riproducibilità}: YARN permette di bloccare le versioni delle dipendenze, questo permette di avere un ambiente di sviluppo riproducibile.
\end{itemize}

\subsubsection*{Svantaggi}
Tra gli svantaggi di YARN troviamo:
\begin{itemize}
    \item \textbf{Dimensioni}: il binario di YARN è molto più grande rispetto a quello di npm;
\end{itemize}


\subsection*{Axios}

\subsubsection*{Descrizione}
Axios è una libreria JavaScript che permette di effettuare richieste HTTP da Node.js o da browser.

\subsubsection*{Vantaggi}
Tra i vantaggi di Axios troviamo:
\begin{itemize}
    \item \textbf{Semplicità d'uso}: Axios fornisce un'API semplice e intuitiva per effettuare richieste HTTP, sia GET che POST, che si integra facilmente con il codice JavaScript esistente;
    \item \textbf{Supporto alle Promise}: utilizza le Promise per la gestione delle richieste asincrone, il che semplifica notevolmente la gestione del flusso di dati e delle operazioni.
\end{itemize}

\subsubsection*{Svantaggi}
\begin{itemize}
    \item \textbf{Dimensioni}: Axios è una libreria molto grande, questo può portare ad un aumento delle dimensioni del bundle finale;
\end{itemize}


\subsection*{CASL}

\subsubsection*{Descrizione}
CASL è una libreria JavaScript che permette di gestire i permessi in maniera semplice
e dichiarativa. È quindi utilizzata per gestire il controllo degli accessi e l'autorizzazione in applicazioni web, attraverso la definzione di regole d'accesso dinamiche e controllare chi ha il permesso di eseguire determinate azioni all'interno dell'applicazione.

\subsubsection*{Vantaggi}
Tra i vantaggi di CASL troviamo:
\begin{itemize}
    \item \textbf{Controllo Fine-Grained degli Accessi}: consente di definire regole di accesso molto precise;
    \item \textbf{integrazione}: si integra facilmente con diverse tecnologie e framework, come ad esempio React e Node.js;
    \item \textbf{Dinamic policies}: permette di definire regole di accesso dinamiche, che possono cambiare a seconda del contesto;
\end{itemize}

\subsubsection*{Svantaggi}
Tra gli svantaggi di CASL troviamo:
\begin{itemize}
    \item \textbf{Curva di apprendimento}: CASL è una libreria molto ampia e complessa, questo può portare ad una curva di apprendimento molto ripida, soprattutto per chi non è avezzo al controllo degli accessi;
    \item \textbf{Complessità aggiuntiva}: l'integrazione e lo sviluppo di applicazioni che impiegano CASL tende ad essere più complesso.
\end{itemize}


\subsection*{csv-parser}

\subsubsection*{Descrizione}
csv-parser è una libreria JavaScript che permette di leggere e parsare file CSV.

\subsubsection*{Vantaggi}
Tra i vantaggi di csv-parser troviamo:
\begin{itemize}
    \item \textbf{Semplicità}: è progettato per essere molto semplice da utilizzare;
    \item \textbf{Dimensioni}: è una libreria molto piccola.
\end{itemize}

\subsubsection*{Svantaggi}
Tra gli svantaggi di csv-parser troviamo:
\begin{itemize}
    \item \textbf{Limitato}: è una libreria molto limitata, infatti non contiene un numero elavato di features.
\end{itemize}


