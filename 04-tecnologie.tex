\section*{Tecnologie}
Nella seguente sezione verranno analizzate le principali tecnologie utilizzate per lo sviluppo del progetto.
Per ogni tecnologia verrà fornita una breve descrizione e verranno elencati i vantaggi e gli svantaggi che hanno portato alla scelta o meno di utilizzarla.
La scelta delle tecnologie è stata fatta tenendo conto delle conoscenze pregresse dei membri del gruppo e della loro disponibilità a imparare nuove tecnologie, così
come della loro popolarità e della loro adozione in ambito aziendale. Inoltre, le scelte dello stack tecnologico hanno tenuto conto
dei requisiti che ambo gli stakeholders (ricordiamo la dualità di questo progetto, università e azienda) hanno espresso.

\subsection*{TypeScript}

\subsubsection*{Descrizione}
TypeScript è un linguaggio di programmazione open-source sviluppato da Microsoft.
È un super-set di JavaScript, ovvero un linguaggio che estende le funzionalità di un altro linguaggio, in questo caso JavaScript.

\subsubsection*{Vantaggi}
Tra i vantaggi di TypeScript troviamo:
\begin{itemize}
    \item \textbf{TypeScript è JavaScript}: essendo un super-set di JavaScript, TypeScript è compatibile con tutte le librerie JavaScript esistenti;
    \item \textbf{TypeScript è un linguaggio fortemente tipizzato}: questo permette di avere un codice più robusto e di trovare più facilmente errori durante la fase di sviluppo;
    \item \textbf{TypeScript è un linguaggio orientato agli oggetti}: questo permette di avere un codice più modulare e di avere una maggiore astrazione rispetto a JavaScript;
\end{itemize}

\subsubsection*{Svantaggi}
Tra gli svantaggi di TypeScript troviamo:
\begin{itemize}
    \item \textbf{TypeScript è un linguaggio fortemente tipizzato}: l'altro lato della medaglia è che il programmatore deve dichiarare il tipo di ogni variabile, questo può portare ad un aumento della verbosità del codice;
\end{itemize}

