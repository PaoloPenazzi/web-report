\section*{Codice}

\subsection*{Credentials}
Il Credentials schema include le funzioni per salare la password e per verificare la correttezza della password inserita dall'utente in fase di login.
\begin{verbatim}
  credentialsSchema.pre<ICredentials>('save', function (next) {
  const credentials = this as ICredentials;

  // only hash the password if it has been modified (or is new)
  if (!credentials.isModified('password')) return next();

  // generate a salt
  bcrypt.genSalt(SALT_WORK_FACTOR, function (err, salt) {
    if (err) return next(err);

    // hash the password using our new salt
    bcrypt.hash(credentials.password, salt, function (err, hash) {
      if (err) return next(err);

      // override the cleartext password with the hashed one
      credentials.password = hash;
      next();
    });
  });
  });

credentialsSchema.methods.comparePassword = function (candidatePassword: string, cb: (err: Error | null, isMatch?: boolean) => void) {
  bcrypt.compare(candidatePassword, this.password, function (err, result) {
    if (err) return cb(err);
    cb(null, result === true);
  });
};

const Credentials = models.Credentials || model('Credentials', credentialsSchema, 'credentials');

export type { ICredentialsData };
export { Credentials };

\end{verbatim}


\subsection*{Esempio di definizione API NextJS}
Di seguito il file pages/api/data/get.ts, che definisce l'API per ottenere i dati da visualizzare nelle dashboard. L'API verrà automaticamente esposta da NextJS alla route /api/data/get e il codice dell'handler verrà automaticamente eseguito lato server.

\begin{verbatim}
import type { NextApiRequest, NextApiResponse } from "next/types";
import { Data } from "../../../@db/schemas/data"
import connectToDatabase from "../../../@db/database"

connectToDatabase()

export default async function handler(req: NextApiRequest, res: NextApiResponse) {
    if (req.method === 'GET') {
        try {
            const { objective, username } = req.query;
            const data = await Data.find({ objective: objective, username: username });
            console.log('[SERVER] found data: \n' + data);
            res.status(200).json(data);
        } catch (error) {
            res.status(500).json({ error: 'Failed to retrieve data from the database' });
        }
    } else {
      
        res.status(400).json({ error: 'Invalid request method' });
    }
}
\end{verbatim}


Per rendere le API accessibili solo agli utenti loggati, è stato utilizzato il middleware di NextJS, che permette di eseguire del codice prima di eseguire l'handler dell'API. Il middleware controlla che l'utente sia loggato, e in caso contrario restituisce un errore 401 (Unauthorized).
In futuro, questo middleware dovrà anche implementare una verifica che l'utente sia effettivamente autorizzato ad accedere ai dati richiesti (authorization).

\begin{verbatim}
import { NextApiRequest, NextApiResponse } from "next/types";
import jwt from "jsonwebtoken";

const jwtConfig = {
  secret: process.env.NEXT_PUBLIC_JWT_SECRET as string,
};

export const authMiddleware = (handler: (
  req: NextApiRequest,
  res: NextApiResponse
) => Promise<void>) => async (
  req: NextApiRequest,
  res: NextApiResponse
): Promise<void> => {
  try {
    const token = req.headers.authorization?.replace("Bearer ", "");

    if (!token) {
      return res.status(401).json({ error: "Authorization token not provided" });
    }

    jwt.verify(token, jwtConfig.secret, (err, decoded) => {
      if (err) {
        return res.status(401).json({ error: "Invalid authorization token" });
      }

      console.log(decoded)

      // If the token is valid, call the original API handler
      return handler(req, res);
    });
  } catch (error) {
    res.status(500).json({ error: "An error occurred during authentication" });
  }
};
\end{verbatim}


\subsection*{Autenticazione}

Per ottenere un codice più pulito, è stato fatto buon uso dei Context di React. Un ottimo esempio è l'AuthContext.tsx, che permette di accedere alle informazioni sulla sessione attuale a tutti i componenti figli di AuthProvider.

\begin{verbatim}
  const defaultProvider: AuthValuesType = {
  user: null,
  loading: true,
  setUser: () => null,
  setLoading: () => Boolean,
  login: () => Promise.resolve(),
  logout: () => Promise.resolve()
}

const AuthContext = createContext(defaultProvider)

type Props = {
  children: ReactNode
}

const AuthProvider = ({ children }: Props) => {
  // ** States
  const [user, setUser] = useState<UserDataType | null>(defaultProvider.user)
  const [loading, setLoading] = useState<boolean>(defaultProvider.loading)

  // ** Hooks
  const router = useRouter()

  useEffect(() => {
    const initAuth = async (): Promise<void> => {
      const storedToken = window.localStorage.getItem(authConfig.storageTokenKeyName)!
      if (storedToken) {
        setLoading(true)
        await axios
          .get(authConfig.meEndpoint, {
            headers: {
              Authorization: storedToken
            }
          })
          .then(async response => {
            setLoading(false)
            setUser({ ...response.data.userData })

             // Set access token in localStorage
             window.localStorage.setItem(authConfig.storageTokenKeyName, response.data.accessToken);
          })
          .catch(() => {
            localStorage.removeItem('userData')
            localStorage.removeItem('refreshToken')
            localStorage.removeItem('accessToken')
            setUser(null)
            setLoading(false)
            if (authConfig.onTokenExpiration === 'logout' && !router.pathname.includes('login')) {
              router.replace('/login')
            }
          })
      } else {
        setLoading(false)
      }
    }

    initAuth()
    // eslint-disable-next-line react-hooks/exhaustive-deps
  }, [])

  const handleLogin = (params: LoginParams, errorCallback?: ErrCallbackType) => {
    axios
      .post(authConfig.loginEndpoint, params)
      .then(async response => {
        params.rememberMe
          ? window.localStorage.setItem(authConfig.storageTokenKeyName, response.data.accessToken)
          : null
        const returnUrl = router.query.returnUrl

        setUser({ ...response.data.userData })
        params.rememberMe ? window.localStorage.setItem('userData', JSON.stringify(response.data.userData)) : null

        const redirectURL = returnUrl && returnUrl !== '/' ? returnUrl : '/'

        router.replace(redirectURL as string)
      })

      .catch(err => {
        if (errorCallback) errorCallback(err)
      })
  }

  const handleLogout = () => {
    setUser(null)
    window.localStorage.removeItem('userData')
    window.localStorage.removeItem(authConfig.storageTokenKeyName)
    router.push('/login')
  }

  const values = {
    user,
    loading,
    setUser,
    setLoading,
    login: handleLogin,
    logout: handleLogout
  }

  return <AuthContext.Provider value={values}>{children}</AuthContext.Provider>
}

export { AuthContext, AuthProvider }
\end{verbatim}




