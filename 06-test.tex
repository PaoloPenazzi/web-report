\section*{Test}
Nella sezione seguente, verranno descritti i test condotti sul prodotto.\\

Inizialmente, le funzionalità del sistema sono state valutate dai
membri del team utilizzando i browser più diffusi, tra cui Chrome e
Firefox; in seguito, il portale è stato testato anche dagli utenti finali ovvero coloro che lo sfrutteranno nel day-by-day. L'obiettivo principale di questa fase di test era garantire la portabilità del sistema.
A tal fine, sono stati eseguiti test su diverse piattaforme, inclusi dispositivi mobili. I test si sono concentrati sulla verifica che le funzionalità principali e i compiti fondamentali funzionassero in modo uniforme su tutti i browser e dispositivi. Inoltre, è stata prestata particolare attenzione alla corretta e comprensibile visualizzazione dei dati. Uno dei requisiti chiave del sistema è il monitoraggio dei dati, e questo aspetto doveva essere gestito in modo semplice e intuitivo.\\
\subsection*{Usability Test}
Durante la fase di verifica del prodotto, il gruppo ha deciso di effettuare
un test di usabilità per valutare la qualità del prodotto.\\

Il prodotto è stato dapprima testato dai membri del gruppo, che hanno
navigato in totale autonomia all'interno del sito, cercando di svolgere 
le operazioni più comuni. In particolare sono state testate:
\begin{itemize}
    \item la navigazione all'interno del sito;
    \item le operazioni di login e logout;
    \item la visualizzazione delle dashboard;
    \item l'upload dei dati (file CSV).
\end{itemize}
Successivamente, il prodotto è stato testato dai destinatari del progetto, i quali hanno riprodotto i medesimi
test.\\

I risultati ottenuti sono stati soddisfacenti. Tutti i membri del team hanno
trovato ragionevole la navigabilità del sistema così come la visualizzazione e 
il rendimento grafico.\\
Al secondo gruppo di test, ovvero i fruitori reali dell'applicazione, non erano stati forniti particolari dettagli sul funzionamento del sistema, ma
è stato detto loro di navigare liberamente all'interno del sito e di svolgere le operazioni che ritenevano più comuni.\\

In sintesi, tutti hanno trovato la disposizione degli elementi all'interno del sistema sempre adeguata e funzionale.
Sono state sollevate alcune osservazioni riguardo a meri dettagli stilistici, nessuno di questi però andava a ledere l'usability del sito.

\subsection*{Accessibilità, Performance e Ottimizzazione SEO}
Per quanto riguarda il testing di accessibilità, performance e ottimizzazione SEO (sebbene fosse poco rilevante per il progetto),
è stato impiegato il tool \textbf{Lighthouse} di Google. In breve, 
è un tool gratuito per monitorare e ottimizzare le performance online, dal punto di vista della velocità e dell'user experience.\\  

Anche in questo caso, il prodotto ha ottenuto risultati soddisfacenti. Dal punto di
vista dell'accessibilità e dell'ottimizzazione SEO non sono stati riscontrati problemi, sono
stati ottenuti punteggi molto alti sia nella versione desktop che nella versione mobile.
Per quanto riguarda le performance, i punteggi ottenuti sono stati molto buoni con 
un leggero calo nella versione mobile.\\

Nota di merito, il sito si è rivelato molto accessibile e perfomante
anche nel caso in cui le dashboard contenessero un numero elevato di dati.\\
