\section*{Requisiti}

L'azienda utilizza attualmente un semplice foglio excel, con una vista alquanto confusionaria, e senza la possibilità per ciascun dipendente di vedere solo i propri dati.

Il software deve permettere di definire obiettivi per l'anno corrente per i dipendenti di un'azienda, e di monitorarne il raggiungimento.

Un obiettivo è definito in questo modo:
\begin{itemize}
    \item Nome, univoco e significativo
    \item Tipologia, gli obiettivi possono essere "Aziendale" (comune a tutti, come il fatturato), "Funzione" (relativo allo specifico ruolo del dipendente), o "Complementare" (un obiettivo raggiungibile dal ruolo del dipendente in collaborazione con altri dipartimenti aziendali), o "Soft Skills" (obiettivi di crescita personale).
    \item Base, il valore di partenza dell'obiettivo.
    \item 100\%, il valore che l'obiettivo deve raggiungere entro fine anno per ottenere il minimo dei punti relativi a quest'obiettivo.
    \item 150\%, il valore che l'obiettivo deve raggiungere entro fine anno per ottenere il massimo dei punti relativi a quest'obiettivo.
    \item Peso, il peso dell'obiettivo rispetto agli altri obiettivi. Gli gli obiettivi assegnati ad un dipendente deve sommare a 100.
    \item Uptodate, il valore attuale dell'obiettivo (i dati possono venire aggiornati man mano durante l'anno).
\end{itemize}

I dipendenti sono modellati in maniera molto semplice, con username, password, nome, cognome e ruolo aziendale.
L'applicativo deve permettere al manager (super-user) di:

\begin{itemize}
    \item Creare l'anagrafica dei dipendenti.
    \item Caricare gli obiettivi per l'anno corrente, assegnati a ciascun dipendente.
    \item Caricare i valori up to date, sulla base dei quali verrà calcolato il punteggio.
\end{itemize}

L'applicativo deve permettere al dipendente di:
    \item Vedere i propri obiettivi, e il loro stato di avanzamento.
    \item Simulare come cambierebbe il punteggio se l'obiettivo raggiungesse un certo valore.
\end{itemize}

Entrambe le tipologie di utenti chiaramente devono poter loggarsi nell'applicativo.
Per il momento il software non deve permettere la registrazioni di nuovi utenti, che verranno configurati manualmente.




